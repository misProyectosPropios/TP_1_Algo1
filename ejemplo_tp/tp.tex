\documentclass[10pt,a4paper]{article}

\usepackage[spanish,activeacute,es-tabla]{babel}
\usepackage[utf8]{inputenc}
\usepackage{ifthen}
\usepackage{listings}
\usepackage{dsfont}
\usepackage{subcaption}
\usepackage{amsmath}
\usepackage[strict]{changepage}
\usepackage[top=1cm,bottom=2cm,left=1cm,right=1cm]{geometry}%
\usepackage{color}%
\newcommand{\tocarEspacios}{%
	\addtolength{\leftskip}{3em}%
	\setlength{\parindent}{0em}%
}

% Especificacion de procs

\newcommand{\In}{\textsf{in }}
\newcommand{\Out}{\textsf{out }}
\newcommand{\Inout}{\textsf{inout }}

\newcommand{\encabezadoDeProc}[4]{%
	% Ponemos la palabrita problema en tt
	%  \noindent%
	{\normalfont\bfseries\ttfamily proc}%
	% Ponemos el nombre del problema
	\ %
	{\normalfont\ttfamily #2}%
	\
	% Ponemos los parametros
	(#3)%
	\ifthenelse{\equal{#4}{}}{}{%
		% Por ultimo, va el tipo del resultado
		\ : #4}
}

\newenvironment{proc}[4][res]{%
	
	% El parametro 1 (opcional) es el nombre del resultado
	% El parametro 2 es el nombre del problema
	% El parametro 3 son los parametros
	% El parametro 4 es el tipo del resultado
	% Preambulo del ambiente problema
	% Tenemos que definir los comandos requiere, asegura, modifica y aux
	\newcommand{\requiere}[2][]{%
		{\normalfont\bfseries\ttfamily requiere}%
		\ifthenelse{\equal{##1}{}}{}{\ {\normalfont\ttfamily ##1} :}\ %
		\{\ensuremath{##2}\}%
		{\normalfont\bfseries\,\par}%
	}
	\newcommand{\asegura}[2][]{%
		{\normalfont\bfseries\ttfamily asegura}%
		\ifthenelse{\equal{##1}{}}{}{\ {\normalfont\ttfamily ##1} :}\
		\{\ensuremath{##2}\}%
		{\normalfont\bfseries\,\par}%
	}
	\renewcommand{\aux}[4]{%
		{\normalfont\bfseries\ttfamily aux\ }%
		{\normalfont\ttfamily ##1}%
		\ifthenelse{\equal{##2}{}}{}{\ (##2)}\ : ##3\, = \ensuremath{##4}%
		{\normalfont\bfseries\,;\par}%
	}
	\renewcommand{\pred}[3]{%
		{\normalfont\bfseries\ttfamily pred }%
		{\normalfont\ttfamily ##1}%
		\ifthenelse{\equal{##2}{}}{}{\ (##2) }%
		\{%
		\begin{adjustwidth}{+5em}{}
			\ensuremath{##3}
		\end{adjustwidth}
		\}%
		{\normalfont\bfseries\,\par}%
	}
	
	\newcommand{\res}{#1}
	\vspace{1ex}
	\noindent
	\encabezadoDeProc{#1}{#2}{#3}{#4}
	% Abrimos la llave
	\par%
	\tocarEspacios
}
{
	% Cerramos la llave
	\vspace{1ex}
}

\newcommand{\aux}[4]{%
	{\normalfont\bfseries\ttfamily\noindent aux\ }%
	{\normalfont\ttfamily #1}%
	\ifthenelse{\equal{#2}{}}{}{\ (#2)}\ : #3\, = \ensuremath{#4}%
	{\normalfont\bfseries\,;\par}%
}

\newcommand{\pred}[3]{%
	{\normalfont\bfseries\ttfamily\noindent pred }%
	{\normalfont\ttfamily #1}%
	\ifthenelse{\equal{#2}{}}{}{\ (#2) }%
	\{%
	\begin{adjustwidth}{+2em}{}
		\ensuremath{#3}
	\end{adjustwidth}
	\}%
	{\normalfont\bfseries\,\par}%
}

% Tipos

\newcommand{\nat}{\ensuremath{\mathds{N}}}
\newcommand{\ent}{\ensuremath{\mathds{Z}}}
\newcommand{\float}{\ensuremath{\mathds{R}}}
\newcommand{\bool}{\ensuremath{\mathsf{Bool}}}
\newcommand{\cha}{\ensuremath{\mathsf{Char}}}
\newcommand{\str}{\ensuremath{\mathsf{String}}}

% Logica

\newcommand{\True}{\ensuremath{\mathrm{true}}}
\newcommand{\False}{\ensuremath{\mathrm{false}}}
\newcommand{\Then}{\ensuremath{\rightarrow}}
\newcommand{\Iff}{\ensuremath{\leftrightarrow}}
\newcommand{\implica}{\ensuremath{\longrightarrow}}
\newcommand{\IfThenElse}[3]{\ensuremath{\mathsf{if}\ #1\ \mathsf{then}\ #2\ \mathsf{else}\ #3\ \mathsf{fi}}}
\newcommand{\yLuego}{\land _L}
\newcommand{\oLuego}{\lor _L}
\newcommand{\implicaLuego}{\implica _L}

\newcommand{\cuantificador}[5]{%
	\ensuremath{(#2 #3: #4)\ (%
		\ifthenelse{\equal{#1}{unalinea}}{
			#5
		}{
			$ % exiting math mode
			\begin{adjustwidth}{+2em}{}
				$#5$%
			\end{adjustwidth}%
			$ % entering math mode
		}
		)}
}

\newcommand{\existe}[4][]{%
	\cuantificador{#1}{\exists}{#2}{#3}{#4}
}
\newcommand{\paraTodo}[4][]{%
	\cuantificador{#1}{\forall}{#2}{#3}{#4}
}

%listas

\newcommand{\TLista}[1]{\ensuremath{seq \langle #1\rangle}}
\newcommand{\lvacia}{\ensuremath{[\ ]}}
\newcommand{\lv}{\ensuremath{[\ ]}}
\newcommand{\longitud}[1]{\ensuremath{|#1|}}
\newcommand{\cons}[1]{\ensuremath{\mathsf{addFirst}}(#1)}
\newcommand{\indice}[1]{\ensuremath{\mathsf{indice}}(#1)}
\newcommand{\conc}[1]{\ensuremath{\mathsf{concat}}(#1)}
\newcommand{\cab}[1]{\ensuremath{\mathsf{head}}(#1)}
\newcommand{\cola}[1]{\ensuremath{\mathsf{tail}}(#1)}
\newcommand{\sub}[1]{\ensuremath{\mathsf{subseq}}(#1)}
\newcommand{\en}[1]{\ensuremath{\mathsf{en}}(#1)}
\newcommand{\cuenta}[2]{\mathsf{cuenta}\ensuremath{(#1, #2)}}
\newcommand{\suma}[1]{\mathsf{suma}(#1)}
\newcommand{\twodots}{\ensuremath{\mathrm{..}}}
\newcommand{\masmas}{\ensuremath{++}}
\newcommand{\matriz}[1]{\TLista{\TLista{#1}}}
\newcommand{\seqchar}{\TLista{\cha}}

\renewcommand{\lstlistingname}{Código}
\lstset{% general command to set parameter(s)
	language=Java,
	morekeywords={endif, endwhile, skip},
	basewidth={0.47em,0.40em},
	columns=fixed, fontadjust, resetmargins, xrightmargin=5pt, xleftmargin=15pt,
	flexiblecolumns=false, tabsize=4, breaklines, breakatwhitespace=false, extendedchars=true,
	numbers=left, numberstyle=\tiny, stepnumber=1, numbersep=9pt,
	frame=l, framesep=3pt,
	captionpos=b,
}


\usepackage{caratula} % Version modificada para usar las macros de algo1 de ~> https://github.com/bcardiff/dc-tex


\titulo{Especificaci\'on}
\subtitulo{Especificacion y correctitud en SmallLang}

\fecha{\today}

\materia{Algortimo y estructura de datos}
\grupo{Compilados}

\integrante{Frutos, I\~{n}aki}{74/24}{inakifrutos00@gmail.com}
\integrante{Pucciarelli, Francisco}{802/22}{fpucciarelli@dc.uba.ar}
\integrante{Apellido, Nacho}{758/24}{iberney@dc.uba.ar}
\integrante{Di Scala,Juan}{004/01}{email4@dominio.com}

% Declaramos donde van a estar las figuras
% No es obligatorio, pero suele ser comodo
\graphicspath{{../static/}}

\begin{document}

\maketitle

\section{Preguntas}
\subsection{Especificaci\'on}


    \begin{enumerate}
        \item \textbf{grandesCiudades}: A partir de una lista de ciudades, devuelve aquellas que tienen m\'as de 50.000 habitantes.
        \begin{proc}{grandesCiudades}{\In ciudades : \TLista{Ciudad}} {\TLista{Ciudad}}\end{proc}

        \item \textbf{sumaDeHabitantes}: Por cuestiones de planificaci\'on urbana, las ciudades registran sus habitantes mayores de edad
        por un lado y menores de edad por el otro. Dadas dos listas de ciudades del mismo largo con los mismos nombres, una
        con sus habitantes mayores y otra con sus habitantes menores, este procedimiento debe devolver una lista de ciudades
        con la cantidad total de sus habitantes.

        \begin{proc}{sumaDeHabitantes}{\In menoresDeCiudades : \TLista{Ciudad}, \In mayoresDeCiudades : \TLista{Ciudad}}{\TLista{Ciudad}}\end{proc}

        \item \textbf{hayCamino}: Un mapa de ciudades est\'a conformada por ciudades y caminos que unen a algunas de ellas. A partir de
        este mapa, podemos definir las distancias entre ciudades como una matriz donde cada celda i, j representa la distancia
        entre la ciudad i y la ciudad j (Fig. 2). Una distancia de 0 equivale a no haber camino entre i y j. Notar que la distancia
        de una ciudad hacia s\'i misma es cero y la distancia entre A y B es la misma que entre B y A.


        Dadas dos ciudades y una matriz de distancias, se pide determinar si existe un camino entre ambas ciudades.

        \begin{proc}{hayCamino}{\In distancias : \TLista{\TLista{\ent}}, \In desde: \ent, \In hasta: \ent}{\bool}\end{proc}

        \item \textbf{cantidadCaminosNSaltos}: Dentro del contexto de redes inform\'aticas, nos interesa contar la cantidad de “saltos”
        que realizan los paquetes de datos, donde un salto se define como pasar por un nodo.
        As\'i como definimos la matriz de distancias, podemos definir la matriz de conexi\'on entre nodos, donde cada celda i, j
        tiene un 1 si hay un  \'unico camino a un salto de distancia entre el nodo i y el nodo j, y un 0 en caso contrario. En este
        caso, se trata de una matriz de conexi\'on de orden 1, ya que indica cu\'ales pares de nodos poseen 1 camino entre ellos a
        1 salto de distancia.

        Dada la matriz de conexi\'on de orden 1, este procedimineto debe obtener aquella de orden n que indica cu\'antos caminos
        de n saltos hay entre los distintos nodos. Notar que la multiplicaci\'on de una matriz de conexi\'on de orden 1 consigo
        misma nos da la matriz de conexi\'on de orden 2, y as\'i sucesivamente.

        \begin{proc}{cantidadCaminosNSaltos}{\Inout conexion : \TLista{\TLista{\ent}}, \In n : \ent}{}\end{proc}
        %\begin{proc}{cantidadCaminosNSaltos}{\Inout conexi\'on: \TLista{\TLista{\ent}}, \in n: \ent}{\ent}\end{proc}

        \item \textbf{caminoM\'inimo}: Dada una matriz de distancias, una ciudad de origen y una ciudad de destino, este procedimiento
        debe devolver la lista de ciudades que conforman el camino m\'as corto entre ambas. En caso de no existir un camino,
        se debe devolver una lista vac\'ia

        \begin{proc}{caminoM\'inimo}{\In origen : \ent, \In destino : \ent, \In distancias: \TLista{\TLista{\ent}}}{}\end{proc}
    \end{enumerate}

\subsection{Demostraciones de correctitud}

La funci\'on poblaci\'onTotal recibe una lista de ciudades donde al menos una de ellas es grande (es decir, supera los
50.000 habitantes) y devuelve la cantidad total de habitantes. Dada la siguiente especificaci\'on:

%TODO
%POner procedimiento
\begin{proc} {poblaci\'onTotal}{\In ciudades : \TLista{Ciudad}}{\ent}
\end{proc}


\begin{lstlisting}
res = 0
i = 0
while (i < ciudades.length) do
    res = res + ciudades[i].habitantes 
    i = i + 1
endwhile
\end{lstlisting}

    \begin{enumerate}
        \item Demostrar que la implementaci\'on es correcta con respecto a la especificaci\'on
        \item Demostrar que el valor devuelto es mayor a 50.000.
    \end{enumerate}


%\newpage


%Respuestas del trabajo
\section{Respuestas}

%Respuestas del punto 1
\subsection{Especificaci\'on}

\begin {enumerate}
    %Ejercicio 1.1)
    %TODO
    \item \begin{proc}{grandesCiudades}{\In ciudades : \TLista{Ciudad}} {\TLista{Ciudad}}
        \requiere{\True} 
	    \asegura{ \\
        |res| = cantidadAparicionesMayorA50000(ciudades) \land \\
        \paraTodo[unalinea]{i}{Ciudad}{(i \in res) \implicaLuego (i \geq 50.000)} \\
        }
	    \aux{auxiliar1}{parametros}{tipoRes}{expresion}
	    \pred{pred1}{parametros}{expresion} 
    \end{proc}

    %Ejercicio 1.2)
    %TODO
    \item \begin{proc}{sumaDeHabitantes}{\In menoresDeCiudades : \TLista{Ciudad}, \In mayoresDeCiudades : \TLista{Ciudad}}{\TLista{Ciudad}}
        \requiere{expresionBooleana1}
	    \asegura{expresionBooleana2}
	    \aux{auxiliar1}{parametros}{tipoRes}{expresion}
	    \pred{pred1}{parametros}{expresion} 
    \end{proc}
    %Ejercicio 1.3)
    %TODO
    \item \begin{proc}{hayCamino}{\In distancias : \TLista{\TLista{\ent}}, \In desde: \ent, \In hasta: \ent}{\bool}
        \requiere{\\
            0 \leq desde < |distancias| \\
            0 \leq hasta < |distancias| \\}
	    \asegura{ \\
            res = \true \Rightarrow \existe[unalinea]{variable}{tipo}{algo \yLuego expresion}
            \existe[unalinea]{variable}{tipo}{algo \yLuego expresion} \Rightarrow res = \true
        }
	    \aux{auxiliar1}{parametros}{tipoRes}{expresion}
	    \pred{pred1}{parametros}{expresion} 
    \end{proc}
    %Ejercicio 1.4)
    %TODO
    \item \begin{proc}{cantidadCaminosNSaltos}{\Inout conexion : \TLista{\TLista{\ent}}, \In n : \ent}{}
        \requiere{expresionBooleana1}
	    \asegura{expresionBooleana2}
	    \aux{auxiliar1}{parametros}{tipoRes}{expresion}
	    \pred{pred1}{parametros}{expresion} 
    \end{proc}
    %Ejercicio 1.5)
    %TODO
    \item \begin{proc}{caminoM\'inimo}{\In origen : \ent, \In destino : \ent, \In distancias: \TLista{\TLista{\ent}}}{}
        \requiere{ \\
            0 \leq origen < |distancias| \\
            0 \leq destino < |distancias| \\
            }
	    \asegura{expresionBooleana2}
	    \aux{auxiliar1}{parametros}{tipoRes}{expresion}
	    \pred{pred1}{parametros}{expresion} 
    \end{proc}

\end{enumerate}


%Respuestas del punto 2
\subsection{Especificaci\'on}

\begin {enumerate}
    %TODO
    \item

\end{enumerate}
\end{document}